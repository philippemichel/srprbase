% Article en Minion pro avec bilio & index, glossaires...
%
\documentclass[a4paper, french]{tufte-handout} 
%
\usepackage{graphicx}
\usepackage{rotating}
\usepackage{xcolor} 
\usepackage{array}
\usepackage{amsmath} 
\usepackage{xltxtra} 
\usepackage{marvosym}
\usepackage{multirow}
\usepackage{textcomp}
\usepackage{booktabs}
\usepackage[output-decimal-marker={,}]{siunitx}
\usepackage{xspace}
\usepackage{babel} 
\usepackage{setspace}
\usepackage{hyperref}
%
 %%%%%%%%%%%%%%%%%%%%%%%%%%%%%
\hypersetup{%
colorlinks=true,
pdftitle={},
pdfauthor={Philippe MICHEL},
pdfkeywords={},
unicode
}
%
\title{Base SRPR}
\author{Philippe MICHEL}
% 
\begin{document} 
%

\maketitle

\section*{Technique}
\label{Technique}


La base de données concerne tous les patients admis au SPRP\\ (CH
Carnelle - Portes de l'Oise). Le médecin responsable de l'étude est le
Dr Caroline \textsc{Grant}.

\begin{itemize}
\item Adresse de la base : \texttt{<https://www.docteur-michel.fr/srpr/>}
\item  Login : \texttt{rea}
\item Mot de passe : \texttt{srpr}
\end{itemize}


La base est hébergée par \textsc{PlanetHoster}. Les serveurs sont en
France. Serveurs sous Linux Debian 7.0. Serveur Apache, bases de données
MariaDB 10.3, interface en PHP 7.3.

Les mots de passe, nom \& prénom des patients sont codés pour le
stockage (hash 128 bits). Le formulaire de saisie est en https donc toutes les
saisies sont codées pendant les transferts.

Déclaration à la CNIL sous le no 2210803.

\section*{Contenu}

La base comprend plusieurs tables contenant des informations sur :
\begin{itemize}
\item État-civil, données démographiques. 
\item État de santé avant l'hospitalisation
\item Le séjour en réanimation\marginnote{Tous les patients admis au
    SRPR sortent de réanimation}
\item État à l'admission au SRPR
\item Bilan à J28
\item Bilan à la sortie
\item Rappel à 6 mois \& un an (pas encore débuté)
\end{itemize}

La base se veut exhaustive \& comprend tous les patients admis au SRPR
(sauf opposition) depuis janvier 2018 soit 96 à ce jour. Une
information générale par affiches dans les salles d'attente est donnée
aux familles.
%
%%%%%%%%%%%%%%%%%%%%%%%%%%%
% 
\end{document}
%%% Local Variables: 
 %%% mode: latex %%% TeX-master: t %%% 